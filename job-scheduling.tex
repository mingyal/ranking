
\subsection{Task scheduling}
Consider a cloud service that offers distributed infrastructure to process
jobs, or a modern data processing system that processes queries on large
databases.
In such systems, queries may be passed to an execution planner
which uses statistics, current workloads, and cost models to choose an optimal
plan among a set of candidate execution plans.
These plans are in the form of directed acyclic graphs with edges indicating
task dependencies.
Some metrics for choosing plans include makespan (time to finish executing
the DAG) and resource consumption (e.g., to obtain fairness between queries).
One challenge, however, is that available resources change with time,
so an execution plan deemed optimal before execution can become sub-optimal
during its lifetime~\cite{osdi-qoop}.

With this as setting, one may consider learning a decision rule for
scheduling tasks.
We consider $\X$ to be some set of feature vectors
(extracted from the query, tasks, resources at time of execution,
time-of-day, and so on), and $\Y$ to be the set of preference graphs with
appropriate augmentation depending on the metric of interest. Samples
$S = \{((x_{i,j})_{j=1}^{n_i}, y_i)_{i=1}^m\}$ may come from computing the best DAG
in hindsight and collecting statistics. We want the learning algorithm to
output a DAG that minimizes $\ell$-risk for some suitable loss $\ell$.

If we make the simplification that execution of plans proceed in rounds (e.g.\
MapReduce), ranking techniques can be brought to bear. Specifically,
let the prediction space be
$\Yh = \{f: \bigcup_{i=1}^n \X^n \to \mathbb{Z}_+\}$,
i.e.\ functions which stratifies jobs into rounds. This is no different from
functions that output relevance scores.

\subsubsection{Minimizing Job Completion Time}
Here, we want the scoring function to output as few rounds as possible.
A way to do this is as follows. For $n$ items, let
\begin{align*}
  \Y_n &= \{ (G, \vec{c})
  \mid \text{$G$ is a weighted DAG on $[n]$ and $\vec{c} = c_1, \ldots, c_n$}
  \}.
\end{align*}
The label space is $\Y = \bigcup_{n=1}^\infty \Y_n$. We interpret each $c_i$
to be the cost incurred for scheduling one task past round $i$.

The loss $\ell$ can then be defined as:
\begin{align*}
  \ell(f, (G, \vec{c}))
  = \sum_{j,k: j \neq k} w_{j,k}^G \1_{f(j) \geq f(k)}
  + \sum_{l=1}^n c_l \sum_{j=1}^n \1_{f(j) \geq l}
\end{align*}
In~\cite{shivani-low-rank}, 

\subsection{Minimizing Resource Consumption}
Here, we want the scoring function to allocate 


