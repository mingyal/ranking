
\subsubsection{Mingyang's draft for Minimizing Job Completion Time}

\iffalse
Let consider $5$ items. The true permutation $\sigma^*: \sigma^*(i)=i$. If we have a mapping $m$ with $m(1)=1, m(2,4)=2, m(3,5)=3$, then it has a wrong ranking for pair $(3,4)$. Let $m'$ be the flip mapping. 
\begin{itemize}
	\item $l(m)=y_{24}+y_{42}+y_{35}+y_{53}+y_{34}$
	\item $l(m')=y_{23}+y_{32}+y_{45}+y_{54}$
\end{itemize}
If $y_{34}=1, y_{43}=0$, which is the only edge in the difference, and $y_{24}=y_{42}=0,y_{35}=y_{53}=0$, $y_{23}=y_{32}=1, y_{45}=y_{54}=0$. Then $l(m)=1<l(m')=2$
\fi


\begin{equation}\label{eq:pdloss}
\begin{split}
%\text{Target Loss: }l: \mathcal{Y}\times\mathcal{T}, \text{ with } l((\mathbf{y},\mathbf{c}),  m)=\sum\limits_{i\not=j}y_{ij}1_{\{m(i)\geq m(j)\}}+\sum\limits_{i=1}^rc_i\big(\sum_{j=1}^r1_{\{m(j)\geq i\}}\big)=(\mathbf{y}', \mathbf{c}')\begin{pmatrix} \mathbf{\Phi}_1(m) \\\mathbf{\Phi}_2(m) \\\end{pmatrix} 
\text{Target Loss: }l: \mathcal{Y}\times\mathcal{T}, \text{ with } l((\mathbf{y},\mathbf{c}),  m)=\sum\limits_{i\not=j}y_{ij}1_{\{m(i)\geq m(j)\}}+\sum\limits_{i=1}^rc_i1_{\{(\max_jm(j))\geq i\}}
\end{split}
\end{equation}
The best scheduling function $m^*\in \text{argmin}E[l((\mathbf{y},\mathbf{c}),  m)]$. Since this loss is a sum of weighted indicator functions, which is not convex and differentiable. In order to derive $m^*$, we seek to use surrogate loss. Note that we can rewrite the target loss as $l((\mathbf{y},\mathbf{c}),  m)=(\mathbf{y}', \mathbf{c}')\begin{pmatrix} \mathbf{\Phi}_1(m) \\\mathbf{\Phi}_2(m) \\\end{pmatrix} $, where $\mathbf{\Phi}_1(m)$ is $r(r-1)$ length vector with entries $1_{\{m(i)\geq m(j)\}}$, and  $\mathbf{\Phi}_2(m)$ is $r$ length vector with entries $1_{\{(\max_jm(j))\geq i\}}$.  By Theorem 3 \cite{ramaswamy2013convex}, we have $l$-calibrated surrogate $(\psi_l^*, \text{pred}_l^*)$ , where
\begin{equation*}
\begin{split}
\psi_l^*((\mathbf{y},\mathbf{c}), (\mathbf{u}, \mathbf{v}))=\sum\limits_{i\not=j}^{r}(u_{ij}-y_{ij})^2+\sum\limits_{k=1}^r(v_k-c_k)^2,\quad \text{pred}_l^*(\mathbf{u}, \mathbf{v})\in\text{argmin}_{m\in \mathcal{T}}\bigg(\mathbf{u}'\mathbf{\Phi}_1(m)+\mathbf{v}'\mathbf{\Phi}_2(m)\bigg)
\end{split}
\end{equation*}
This method shows that using simple squared loss, we still can have a consistent solution once we get scores which minimize surrogates and plug these scores into pred function to get $m^*$. Define $y_{ij}^\mathbf{p}=E_\mathbf{p}[y_{ij}]$, $c_{i}^\mathbf{p}=E_\mathbf{p}[c_i]$. Then scores from surrogates are $u_{ij}=y_{ij}^\mathbf{p}$, $v_k=c_k^\mathbf{p}$. However, the optimization problem associated with computing $\text{pred}_l^*(\mathbf{u}, \mathbf{v})$ above is a minimum weighted feedback arc set (MWFAS) problem, which is known to be NP-hard.



Let $m^*: [r]\rightarrow [r]$ be the optimal solution of eq(\ref{eq:pdloss}), and let $|m|=\max_jm(j)$. Since we want to take advantage of ranking, we would hope $m^*$ would have below two properties, which are stated in Lemma~\ref{lemma:1}, Lemma~\ref{lemma:3}.

\begin{lemma}\label{lemma:1}
No gap: For any $k\in[|m|]$, $m^{-1}(k)\not=\emptyset$. 
\end{lemma}
\begin{proof}[Proof of Lemma \ref{lemma:1}]
If there exists such a $k$, then we can move all the items ranked lower than $k$ w.r.t $m$ by one level, i.e, set $m'(l)=m(l)-1, \forall l, m(l)>k$. We can reduce the target loss by $c_{|m|}>0$, which contradicts that $m$ is optimal.
\end{proof}



The second property is Respect DAG: If $\sigma^*(k)<\sigma^*(l)$, then $m(k)\leq m(l)$.


For us, this property is very initiutive. Also it is very important for us to incorporate the ranking result into this scheduling problem.  Unfortunately, this property is not true generally.

\begin{lemma}\label{lemma:2}
The second property does not hold even under low noise condition. 
\end{lemma}
\begin{proof}[Proof of Lemma \ref{lemma:2}]
Consider that a true ranking over $3$ items is $1\rightarrow 2\rightarrow 3$. $y_{12}^\mathbf{p}=10+1e^{-5}, y_{21}^\mathbf{p}=10, y_{23}^\mathbf{p}=10+1e^{-5}, y_{32}^\mathbf{p}=10, y_{13}^\mathbf{p}=3\times1e-5, y_{31}^\mathbf{p}=0, c_3^\mathbf{p}=+\infty, c_1^\mathbf{p}=c_2^\mathbf{p}=1e^{-10}$(satisfying low-noise condition).

Then obviously $|m^*|=2$. Consider 3 cases. The first 2 cases are only cases respecting orders.
	\begin{itemize}
		\item $m_1: 1\rightarrow 2,3$. $l(m)=y_{23}^\mathbf{p}+y_{32}^\mathbf{p}+y_{21}^\mathbf{p}+y_{31}^\mathbf{p}>30$
		\item $m_2: 1, 2\rightarrow 3$. $l(m)=y_{12}^\mathbf{p}+y_{21}^\mathbf{p}+y_{31}^\mathbf{p}+y_{32}^\mathbf{p}>30$
		\item $m_3: 1, 3\rightarrow 2$. $l(m)=y_{13}^\mathbf{p}+y_{31}^\mathbf{p}+y_{21}^\mathbf{p}+y_{23}^\mathbf{p}<30$
	\end{itemize}
The best scheduling function $m^*$ will not respect ranking order. 
\end{proof}

The problem is on the inconsistent magnitutde of perference. In the example, although the difference $y_{13}^\mathbf{p}-y_{31}^\mathbf{p}$ is bigger than the other two, but the magnitude of the perference $y_{13}^\mathbf{p}$ and $y_{31}^\mathbf{p}$, or a more direct measure of consistency of magnitude, $y_{13}+y_{31}$ is not comparable w.r.t $y_{12}^\mathbf{p}+y_{21}^\mathbf{p}$ and $y_{23}^\mathbf{p}+y_{32}^\mathbf{p}$.  So we come up with an assumption:  the clustering cost of two items $y_{ij}+y_{ji}$ is no less than the sum of clustering cost of any two adjacent items w.r.t $\sigma^*$ between $i,j$. Here two adjacent items w.r.t $\sigma^*$ between $i,j$ means $\max(\sigma^*(k),\sigma^*(l))\leq  \max(\sigma^*(i),\sigma^*(j))$ and $\min(\sigma^*(k),\sigma^*(l))\geq  \min(\sigma^*(i),\sigma^*(j))$ and $|\sigma^*(k)-\sigma^*(l)|=1$. The more formal form is stated in below.

\begin{assumption}
	There is a unique total order $\sigma^*$. For any  $i\not=j $, wlog $\sigma^*(i)<\sigma^*(j)$
	$$y_{ij}^\mathbf{p}+y_{ji}^\mathbf{p}\geq \sum\limits_{k\in \{l: \sigma^*(l)\in [\sigma^*(i), \sigma^*(j)]\}, l={(\sigma^*)}^{-1}(\sigma^*(k)+1)\}}y_{kl}^\mathbf{p}+y_{lk}^\mathbf{p}$$
\end{assumption}


To justify this assumption, let's consider another appealing ( perference stability )assumption 2: if for one $G\in \mathcal{G}$, $y_{ij}\geq 0$, then $y_{ij}\geq0$ for the rest of graphs. That implies, at least one of $y_{ij}^\mathbf{p}$ and $y_{ji}^\mathbf{p}$ is $0$.  Under this assumption,  Assumption 1 holds automatically under low noise conidtion by following argument. Assume there is a path $i_1\rightarrow...\rightarrow i_k$ in the difference graph, that is to say,
\begin{itemize}
	\item $(y_{i_1,i_2}^\mathbf{p}-y_{i_2,i_1}^\mathbf{p}),...(y_{i_{k-1},i_k}^\mathbf{p}-y_{i_{k}, i_{k-1}}^\mathbf{p})>0$ by uniqueness of rank;
	\item $y_{i_2,i_1}^\mathbf{p} = ... =y_{i_{k}, i_{k-1}}^\mathbf{p} = 0$ by assumption 2.
	\item  $y_{i_1, i_k}^\mathbf{p}-y_{i_k,i_{k-1}}^\mathbf{p}\geq (y_{i_1,i_2}^\mathbf{p}-y_{i_2,i_1}^\mathbf{p})+...+(y_{i_{k-1},i_k}^\mathbf{p}-y_{i_{k}, i_{k-1}}^\mathbf{p})>0$ by low noise
\end{itemize} Then it is just equivalent to  
$y_{i_1, i_k}^\mathbf{p}+y_{i_k,i_{k-1}}^\mathbf{p}\geq (y_{i_1,i_2}^\mathbf{p}+y_{i_2,i_1}^\mathbf{p})+...+(y_{i_{k-1},i_k}^\mathbf{p}+y_{i_{k}, i_{k-1}}^\mathbf{p})>0$
which is just assumption 1. 

Now we want to show Property 2 under assumption 1. 	 Before the proof, we first re-write the target loss as three parts: pdloss, clusterloss, depthloss$$\sum\limits_{i\not=j}y_{ij}1_{\{m(i)> m(j)\}}+\sum\limits_{i\not=j}y_{ij}1_{\{m(i)= m(j)\}}+\sum\limits_{i=1}^rc_i1_{\{(\max_jm(j))\geq i\}}$$

\begin{lemma}
  \label{lemma:3}
	Respect DAG: If $\sigma^*(k)<\sigma^*(l)$, then $m(k)\leq m(l)$ under assumption 1.
\end{lemma}
\begin{proof}

	
	
	
	If not, then for $m$, there must exist two adjacent clusters which do not respect DAG. That is to say, for some $l,k$ with $l=k+1$, and let the $k$-th cluster $C_1=\{i: m(i)=k\}$, the $l$-th cluster $C_2=\{j:m(j)=l\}$, $\exists \text{ pair } (i,j)\in (C_1, C_2)$, $\sigma^*(i)>\sigma^*(j)$. 
	
	If there do not exist such two adjacent clusters, then it means all items in $i$-th cluster ranks higher than $i+1$-th cluster for any $i$. This just means, $m$ respect the ranking order.
	
	
	
	Now consider $m'$ which flip all these pairs of $m$ until we can not find more such pairs. and keep the rest of clusters unchanged. When we cannot find more such pairs, let $C_{1,u}$ be items previously in $C_1$ and now still in $C_1$;   $C_{1, d}$ be items previously in $C_1$ but now in $C_2$;  et $C_{2,u}$ be items previously in $C_2$ but now in $C_1$;   $C_{2, d}$ be items previously in $C_2$ and now still in $C_2$. Since we only do flip pairs, $|C_{1,d}|=|C_{2,u}|$.
	\begin{equation}
	\begin{split}
	&C_1=  C_{1,u}\cup C_{1,d}\\
		&C_2=  C_{2,u}\cup C_{2,d}\\
		&m: m(C_1)=k; m(C_2)=l\\
		&m': m'(C_{1, u}\cup C_{2, u})=k; m'(C_{1,d}\cup C_{2, d})=l\\
	\end{split}
	\end{equation}
	\begin{itemize}
		\item pdloss: $m'$ will have no greater pdloss than $m$, because $m'$ respect more ranking than $m$.
		\item depthloss: $m$ and $m'$ have same depthloss.
		\item clusteringloss: let $C_{cl}(m)$, $C_{cl}(m')$ be clusteringloss for $m, m'$
		\begin{itemize}
			\item $C_{cl}(m)=\bigg(\sum\limits_{i\in C_{1, u}, j\in C_{1, u}}y_{ij}^\mathbf{p} + \sum\limits_{i\in C_{2, d}, j\in C_{2, d}}y_{ij}^\mathbf{p} \bigg)+ \bigg(\sum\limits_{i\in C_{1, u}, j\in C_{1, d}}(y_{ij}^\mathbf{p}+y_{ji}^\mathbf{p}) \bigg)+\bigg( \sum\limits_{i\in C_{2, u}, j\in C_{2, d}}(y_{ij}^\mathbf{p}+y_{ij}^\mathbf{p})\bigg)$
						\item $C_{cl}(m')=\bigg(\sum\limits_{i\in C_{1, u}, j\in C_{1, u}}y_{ij}^\mathbf{p} + \sum\limits_{i\in C_{2, d}, j\in C_{2, d}}y_{ij}^\mathbf{p} \bigg)+ \bigg(\sum\limits_{i\in C_{1, u}, j\in C_{2, u}}(y_{ij}^\mathbf{p}+ y_{ij}^\mathbf{p})\bigg)+\bigg( \sum\limits_{i\in C_{1, d}, j\in C_{2, d}}(y_{ij}^\mathbf{p}+y_{ij}^\mathbf{p})\bigg)$
		\end{itemize}
	By our flip rule, $\sigma^*(i)>\sigma^*(j)$ for any $i\in C_{1,d}\cup C_{2,d}, j\in C_{2,u}\cup C_{1,u}$. If not, the fliping process will not stop. Since the first big bracket is same for $m$ and $m'$, we only work on the rest two. For any $i\in C_{1,u}$, $j\in C_{2,d}, k\in C_{2,u}, l\in C_{1,d} $, There are four possibilities:
	\begin{itemize}
		\item $\sigma^*(k)<\sigma^*(i)<\sigma^*(j)<\sigma^*(l)$
			\item $\sigma^*(k)<\sigma^*(i)<\sigma^*(l)<\sigma^*(j)$
				\item $\sigma^*(i)<\sigma^*(k)<\sigma^*(j)<\sigma^*(l)$
					\item $\sigma^*(i)<\sigma^*(k)<\sigma^*(l)<\sigma^*(j)$
	\end{itemize}
For the first case
\begin{equation*}
\begin{split}
(y_{il}^\mathbf{p}+y_{li}^\mathbf{p})+(y_{kj}^\mathbf{p}+y_{jk}^\mathbf{p})&\geq (y_{ij}^\mathbf{p}+y_{ji}^\mathbf{p}+y_{jl}^\mathbf{p}+y_{li}^\mathbf{p}) + (y_{ki}^\mathbf{p}+y_{ik}^\mathbf{p}+y_{ij}^\mathbf{p}+y_{ji}^\mathbf{p})\\
&>y_{ik}^\mathbf{p}+y_{ki}^\mathbf{p}+y_{jl}^\mathbf{p}+y_{lj}^\mathbf{p}\\
\end{split}
\end{equation*}
where the last strict inequality follows from uniqueness of total order and low noise conditions. Uniqueness implies for any adjacent pair $(i,j)$ w.r.t $\sigma^*$( a.k.a  $|\sigma^*(i)-\sigma^*(j)|=1$), then $|y_{ij}^\mathbf{p}-y_{ji}^\mathbf{p}|>0$. Low noise condition implies for any pair $(k,l)$, wlog $\sigma^*(k)-\sigma^*(l)<-1$, then $y_{kl}^\mathbf{p}-y_{lk}^\mathbf{p}>\sum\limits_{\text{adjacent pairs }(i,j) \text{cbetween w.r.t }\sigma^* }y_{ij}^\mathbf{p}-y_{ji}^\mathbf{p}>0$. This further implies $y_{kl}^\mathbf{p}+y_{lk}^\mathbf{p}>0$ for any pairs. 

For the second case
\begin{equation*}
\begin{split}
(y_{il}^\mathbf{p}+y_{li}^\mathbf{p})+(y_{kj}^\mathbf{p}+y_{jk}^\mathbf{p})&\geq (y_{il}^\mathbf{p}+y_{li}^\mathbf{p}) + (y_{ki}^\mathbf{p}+y_{ik}^\mathbf{p}+y_{ij}^\mathbf{p}+y_{ji}^\mathbf{p}+y_{lj}^\mathbf{p}+y_{jl}^\mathbf{p})\\
&>y_{ik}^\mathbf{p}+y_{ki}^\mathbf{p}+y_{jl}^\mathbf{p}+y_{lj}^\mathbf{p}\\
\end{split}
\end{equation*}
For the third case
\begin{equation*}
\begin{split}
(y_{il}^\mathbf{p}+y_{li}^\mathbf{p})+(y_{kj}^\mathbf{p}+y_{jk}^\mathbf{p})&\geq(y_{kj}^\mathbf{p}+y_{jk}^\mathbf{p}) + (y_{ki}^\mathbf{p}+y_{ik}^\mathbf{p}+y_{kj}^\mathbf{p}+y_{jk}^\mathbf{p}+y_{lj}^\mathbf{p}+y_{jl}^\mathbf{p})\\
&>y_{ik}^\mathbf{p}+y_{ki}^\mathbf{p}+y_{jl}^\mathbf{p}+y_{lj}^\mathbf{p}\\
\end{split}
\end{equation*}
For the fourth case
\begin{equation*}
\begin{split}
(y_{il}^\mathbf{p}+y_{li}^\mathbf{p})+(y_{kj}^\mathbf{p}+y_{jk}^\mathbf{p})&\geq (y_{kl}^\mathbf{p}+y_{lk}^\mathbf{p}+y_{jl}^\mathbf{p}+y_{li})^\mathbf{p} + (y_{ki}^\mathbf{p}+y_{ik}^\mathbf{p}+y_{kl}^\mathbf{p}+y_{lk}^\mathbf{p})\\
&>y_{ik}^\mathbf{p}+y_{ki}^\mathbf{p}+y_{jl}^\mathbf{p}+y_{lj}^\mathbf{p}\\
\end{split}
\end{equation*}
	\end{itemize}
Above calculation shows $C_{cl}(m)>C_{cl}(m')$ which contradicts optimality of $m$.
	
\end{proof}





\begin{lemma}
	The optimal $m^*\in\mathcal{M}$, where  $\mathcal{M}$ is a set containing all elements satisfying Fact2.
\end{lemma}
\iffalse
Note that if there is a \underline{unique} total order $\sigma^*$, then $\mathcal{M}=\{\sigma^*\}$, and $m$ can be derived by clustering items in $m^*_{pd}$. This motivates us a naive algorithm for deriving $m^*$. Basically, we start from $\sigma^*$, and find the best way to cluster items, that is, reduce the target loss by 
$$\textbf{ Increase SMALL loss (from $\mathbf{y}$)by clustering, but decrease BIG loss (from $\mathbf{c}$) by smaller depth }$$

We also notice that minimizing eq(\ref{eq:pdloss}) is equivalent to minimize the following quantity:
	\begin{equation*}
	\text{argmin}_ml((\mathbf{y},\mathbf{c}),  m)= \text{argmin}_m\sum\limits_{i=1}^r\sum\limits_{j=1}^{i-1}(y_{ij}-y_{ji})1_{\{m(i)> m(j)\}}+\sum\limits_{i=1}^r\sum\limits_{j=1}^{i-1}y_{ij}1_{\{m(i)= m(j)\}}+\sum\limits_{i=1}^rc_i1_{\{(\max_jm(j))\geq i\}}\\
	\end{equation*} 

Before we present first algorithm, we start from some notations. For any $m\in \mathcal{M}$, by fact1, we know $m^{-1}(i)\not=\emptyset$ for $i\leq |m|$. 
\begin{itemize}
	\item Let $\mathcal{C}^m=\{  \{m^{-1}(i)\}_{i=1}^{|m|}     \}$ be set of clusters of $m$
	\begin{itemize}
		\item $\mathcal{C}^m[i]= \{m^{-1}(i)\}$ is set of items in cluster $i$ under $m$
		\item Note that $\sigma^*$ has $r$ clusters, and each has one  and only one item.
	\end{itemize}
\item Let cost of merging cluster $i$ and $i+1$ under $m$ be \begin{equation}\label{eq:cost}
\begin{split}
cost^m_{i, i+1}&=\sum\limits_{n_1\in \mathcal{C}^m[i]}\sum\limits_{n_2\in \mathcal{C}^m[i+1]}\bigg(-(y_{n_1n_2}^\mathbf{p}-y_{n_2n_1}^\mathbf{p})1_{n_1>n_2}1_{\{m(n_1)> m(n_2)\}}+y_{n_1n_2}^\mathbf{p}1_{n_1>n_2}+y_{n_2n_1}^\mathbf{p}1_{n_2>n_1}\bigg)\\
\end{split}
\end{equation}
, where each summand is non-negative.
\end{itemize}

	We first derive a consistent permutation $\sigma^*$ under low.noise condition(or related conditions) \cite{ramaswamy2013convex}, \cite{duchi2011}, and then plug $\sigma^*$ into below algorithm. If there is a unique total ranking and it satisfies low.noise assumptions, we will find a $m$ achieving lowest target loss.
	
	\begin{lemma}
	Given $m$,  define $cost=\text{min}_{i\in [|m|-1]} cost^m_{i, i+1}$. $cost$ is non-decreasing after each clustering.
	\end{lemma}
\begin{proof}
	Consider we have $m$, and then we cluster $i, i+1$ to have $m'$. We need to show $\text{min}_{l\in [|m|-1]} cost^m_{l, l+1}\leq \text{min}_{l'\in [|m'|-1]} cost^{m'}_{l', l'+1}$. Notice that $cost^m_{l, l+1}=cost^{m'}_{l', l'+1}$ for $l=l'+1$ and $l\not= i, i+1$. Also $cost^m_{i-1, i}\leq cost^{m'}_{i-1, i}$ because now there are more items in $i$-th cluster under $m'$ and each summand  of eq(\ref{eq:cost}) is non-negative. Same reason for $cost^m_{i, i+1}\leq cost^{m'}_{i, i+1}$. 
	\end{proof}
Thanks to the fact that $c_i$ is strictly increasing, this lemma implies once $c_{cur}-cost$ is negative, any kind of clustering in the future will increase target loss. This justifies the stopping condition in the algorithm. 

\begin{algorithm}[H]
	\caption{Naive algorithm}
	\begin{algorithmic}[1]
		\Procedure{Clustering ranking }{$\mathcal{G}, \mathbf{p}, \mathbf{c}, \sigma$}
		\State Initialize $cur=r$
		\For{\texttt{$i\in[r-1]$}}
		\State
		Let $cost=\text{min}_{i\in [|m|-1]} cost^m_{i, i+1}$
		\State 
		Let $cl = \text{argmin}_{i\in [|m|-1]} cost^m_{i, i+1}$
		\If{$cost\leq c_{cur}$}
		\State Merge clusters $cl$ and $cl+1$. That is, 
		set $m^{-1}(cl)\leftarrow m^{-1}(cl)\cup m^{-1}(cl+1)$; set $m^{-1}(cl+i)\leftarrow m^{-1}(cl+i+1)$ for $i\geq 1$; and set $|m|\leftarrow |m|-1$.
		\State $cur\leftarrow cur-1$
		\Else\quad
			break; 
		\EndIf		
		\EndFor
		\State \textbf{return} $m$
		\EndProcedure
	\end{algorithmic}
\end{algorithm}
The last thing we should worry is that when there are more than $1$ permutation achieve lowest pd loss. We should note that we may only derive one permutation from algorithm in \cite{ramaswamy2013convex}, so whether other unoutputed permutation will achieve lower target loss?  The answer is YES. Let's look at an example.
\begin{figure}[h]
	\begin{center}
\includegraphics[scale=0.5]{figure/eg1.png}
	\end{center}
\end{figure}
The total ranking is not unique. $\sigma_1=(1,2,3,4,5)$ or $\sigma_2=(1,2,4,3,5)$ both minimize pd loss, but will have different solution in our case. Whether $\sigma_1$ or $\sigma_2$ got outputed depends on $(\mathcal{G}, \mathbf{p})$. If we utilize surrogates from Theorem 6 from \cite{ramaswamy2013convex}, 
\begin{equation*}
\begin{split}
u_2=-y_{12}^{\mathbf{p}}+y_{24}^{\mathbf{p}}\\
u_3 = -y_{13}^{\mathbf{p}}+y_{35}^{\mathbf{p}}\\
u_4 = -y_{24}^{\mathbf{p}}-y_{14}^{\mathbf{p}}\\
u_5 = -y_{35}^{\mathbf{p}}-y_{15}^{\mathbf{p}}\\
\end{split}
\end{equation*}
where according to the graph, $y_{21}^{\mathbf{p}}, y_{42}^{\mathbf{p}}, y_{41}^{\mathbf{p}}, y_{31}^{\mathbf{p}}, y_{51}^{\mathbf{p}}, y_{53}^{\mathbf{p}}=0$. It is easily to see that
\begin{equation*}
\begin{split}
\{y_{12}=100, y_{24}^{\mathbf{p}}=100, y_{13}^{\mathbf{p}}=100, y_{35}^{\mathbf{p}}=99, y_{14}^{\mathbf{p}}=200, y_{15}^{\mathbf{p}}=300\}\Rightarrow \sigma_1\\
\{y_{12}=100, y_{24}^{\mathbf{p}}=100, y_{13}^{\mathbf{p}}=400, y_{35}^{\mathbf{p}}=99, y_{14}^{\mathbf{p}}=200, y_{15}^{\mathbf{p}}=499\}\Rightarrow \sigma_2
\end{split}
\end{equation*}

Although $\sigma_1, \sigma_2$ are equivalent w.r.t pdloss, it is not for our case. Let $c_1=0, c_2=c_3=c_4=c_5=1$. Then if we start from $\sigma_1$, the best $m_1=(1,\{2,3\}, \{4,5\})$ and from $\sigma_2$, the best $m_2=(1,2,\{3,4\}, 5)$. However $m_1$ have smaller target loss. 

To solve this issue, we must go through all possible total ordering. That is, if we have all topological sorting of difference graph, we can run Algorithm 1 on each of them to find out the lowest target loss. Thanks to Theorem 7 of \cite{ramaswamy2013convex}, we will have a calibrated surrogates which will also give us all possible total ordering.

\begin{algorithm}[H]
	\caption{Improved algorithm}
	\begin{algorithmic}[1]
		\Procedure{xxx }{$\mathcal{G}, \mathbf{p}, \mathbf{c}$}
		\State Get $\mathcal{M}$ from Algorithm $1$ from \cite{ramaswamy2013convex}
		\State Let $m^*=NULL$, $loss = \infty$
		\For{\texttt{$\sigma\in\mathcal{M}$}}
		\State
		Let $m=${Clustering ranking }{($\mathcal{G}, \mathbf{p}, \mathbf{c}, \sigma$)}, with $l$ be the corresponding loss.
		\If{\texttt{$l\leq loss$}}
		Set $m^*=m$
		\EndIf
		\EndFor
		\State \textbf{return} $m^*$
		\EndProcedure
	\end{algorithmic}
\end{algorithm}

Warning: this algorithm only works under the assumption that $\mathcal{M}$ is small. If $\mathcal{M}$ is huge, then it would be hard to solve the problem quickly. Consider a DAG over $r$ items($r$ is even), with edges $\{(i,i+1)\}_{i=1}^{r/2}\cup \{(i,i+1)\}_{r/2+1}^r$. Then $|\mathcal{M}|=(\frac{r}{2}+1)!$. 
\fi









