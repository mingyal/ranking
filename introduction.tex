\section{The problem of ranking}
In this project, we study the general problem of supervised ranking and survey
work in the area, paying attention to consistency results and
algorithms that perform empirically well but lack theoretical backing.

Roughly, the goal of supervised ranking is to learn from appropriate samples
in some way so as to learn how to rank, i.e., given a new set of $n$ items,
determine how they can be ordered so as to do well on some performance measure.

The canonical application is web search. Here, the search engine is given
a query $q$ and may find $n$ documents $v_1, \ldots, v_n$, but
has to determine an order in which they should be presented to a user.
Early on, this ordering was determined by hand-tuned models~\cite{bm25},
but there is good reason to believe that models adapting to collected data
(samples) would perform better. 

Some appropriate samples are:
\begin{itemize}
\item (Binary relevance)
  $S = \{ ((q_i, (v_{i, j})_{j=1}^{n_i}), (y_{i, j})_{j=1}^{n_i}) \}_{i=1}^m$, where
  $y_{i,j} \in \{0, 1\}$.
  That is, an instance is a query $q_i$ with a list of $n_i$ documents;
  the label for the instance is a vector $(y_{i, j})_{j=1}^{n_i}$ indicating,
  for each document, whether it is relevant or not.
  This may be determined, e.g., by whether there is a
  user who clicked on the link or not.
\item (Relevance scores)
  $S = \{ ( (q_i, (v_{i, j})_{j=1}^{n_i}), (y_{i, j})_{j=1}^{n_i}) \}_{i=1}^m$,
  where $y_{i,j} \in R \subseteq \R_+$.
  Same as the previous case, except that each document has a relevance
  score. 
  This may be determined, e.g., by the number of users who clicked on the link.
\item (Preference graphs) 
  $S = \{((q_i, (v_{i,j})_{j=1}^{n_i}), G_i)\})_{i=1}^m$, where $G_i$ is a weighted
  directed graph on the $n_i$ documents, all weights being non-negative.
  We can interpret $v_j \overset{w_{j,k}}{\to} v_k$ as saying that $v_j$ is
  preferred to $v_k$ with an importance weight of $w_{j,k}$.
  Note that this case subsumes the first two: a label for binary relevance
  can be coded as a directed bipartite graph in the obvious way
  with all weights $1$, and a label for relevance scores may be coded as a
  DAG with weights being the difference in scores.
  An important special case is when the graph consists
  of just a single edge, expressing a single pairwise preference.
  Preference graphs allow for cases where complete information on relevance is
  hard to obtain; for instance, an e-commerce ranking system may obtain $S$
  from a consumer survey, and surveyees should be allowed to specify
  preferences for only a small set of items. Ranking from preference graphs
  also have applications beyond just information retrieval, e.g., in
  recommendation systems and social choice theory, where we may want to
  aggregate preferences from different agents into a global ranking~\cite{}.
\end{itemize}
One may view relevance scores as a natural multiclass generalization of binary
relevance labels, and preference graphs as the
structured-prediction-generalization of relevance scores.

In each case, the goal is for the learning algorithm to learn from these samples
and output a ranking function
$f: Q \times \bigcup_{n=1}^\infty D^n \to \bigcup_{n=1}^\infty P_n$, where $Q$
is the set of queries, $D$ is the set of documents,
and $P_n$ is the set of permutations on $[n]$.
The formalism can be simplified by thinking of query-document pairs as feature
vectors in some space $\X$ (which is also done in practice), so
that dependence on queries can be dropped.

Formally, the supervised ranking problem can be framed as follows. There is an
instance space $\X$, a label space $\Y$ (which varies as above), and the
prediction space is $\Yh = \bigcup_{n=1}^\infty P_n$. A permutation
$\sigma \in Y_h$ is interpreted such that $\sigma(i)$ is the position of the
$i$th item. A learning algorithm
takes $S = ((x_{i,j})_{j=1}^{n_i}, y_i)_{i=1}^m$ as input,
where $x_{i,j} \in \X \subseteq \R^d$ is a feature vector
and $y_i \in \Y$ is a label following the cases above,
and outputs a function
$f_S: \bigcup_{n=1}^\infty \X^n \to \bigcup_{n=1}^\infty P_n$.
There is no loss of generality, since the feature space can in principle 
include $Q$ itself.

Performance of the learning algorithm is measured by some performance measure
$M: \Yh \times \Y \to \R_+$ or loss function
$\ell: \Yh \times \Y \to \R_+$. The goal is to minimize the $\ell$-risk
\begin{align*}
  \err_D^\ell[f_S] = \E_{(\vec{X}, Y) \sim D} [\ell(f_S(\vec{X}), Y)],
\end{align*}
for all distributions $D$ on $(\bigcup_{n=1}^\infty X^n) \times \Y$
within some suitable class.
The notions of Bayes $\ell$-risk, conditional $\ell$-risk,
Bayes conditional $\ell$-risk, $\ell$-regret, and $\ell$-consistency of the
learning algorithm can all be defined similarly to binary classification
covered in class.

Some common measures and losses are:
\begin{itemize}
\item Precision: For binary relevance labels, 
  \begin{align*}
    \text{AvgPrec}@K(\sigma, (y_j)_{j=1}^n)
    = \frac{1}{K} \sum_{r=1}^K \text{Prec}@r(\sigma, (y_j)_{j=1}^n)
    = \frac{1}{K} \sum_{r=1}^K \frac{\sum_{j : \sigma(j) \leq r} y_j}{r}.
  \end{align*}
  Precision at $r$ is the proportion of relevant documents among those 
  ranked up to $r$ according to $\sigma$. Average precision is the
  non-truncated version of this measure, and mean average precision (MAP)
  averages \text{AvgPrec} over all queries in a set, i.e.
  \begin{align*}
    \text{MAP}(\sigma, ((y_j)_{j=1}^{n_i})_{i=1}^m) =
    \frac{1}{m} \sum_{i=1}^m \text{AvgPrec}(\sigma, (y_j)_{j=1}^{n_i}).
  \end{align*}
\item Normalized Discounted Cumulative Gain (NDCG): For relevance scores,
  the discounted cumulative gain at $K$ is 
  \begin{align*}
    \text{DCG}@K(\sigma, (y_j)_{j=1}^n)
    = \sum_{r=1}^K \frac{G(y_{\sigma^{-1}(r)})}{D(r)}
    = \sum_{i: \sigma(i) \leq K} \frac{G(y_i)}{D(\sigma(i))}.
  \end{align*}
  (The first sum runs over positions and the second sum runs over
  object indices.) The gain function $G$ boosts relevance scores, and is
  normally taken to be $G(y) = 2^y - 1$. The discount function $D$ discounts
  the importance of the term based on the position of the item, and is
  normally taken to be $D(\sigma(i)) = \log(1 + \sigma(i))$. As before, DCG
  is the non-truncated version of this, and NDCG normalizes DCG to obtain
  a number in $[0,1]$, defined as
  \begin{align*}
    \text{NDCG}(\sigma, (y_j)_{j=1}^n)
    = \frac{1}{N((y_j)_{j=1}^n)} DCG(\sigma, (y_j)_{j=1}^n),
  \end{align*}
  where
  \begin{align*}
    N((y_j)_{j=1}^n)
    = \frac{1}{\max_{\sigma'} DCG(\sigma', (y_j)_{j=1}^n)} DCG(\sigma, (y_j)_{j=1}^n).
  \end{align*}
  In many applications, it is more important for the output to rank well for
  items that are near the top compared to ones below, and to rank well for
  highly relevant items compared to those with low scores; NDCG factors
  these in using the discount and gain functions respectively.

\item Weighted pairwise disgreement loss: For preference graphs, the most
  natural loss to define is just:
  \begin{align*}
    \ell_{\text{PD}}(\sigma, G)
    = \sum_{j,k: j \neq k} w_{j,k}^G \1[\sigma(j) > \sigma(k)]
  \end{align*}
  Here the sum is over object indices, and we can take $w_{j,k}^G = 0$ if
  the edge $j \to k$ is not in $G$. 
\end{itemize}
Note that while the pairwise disagreement (PD) loss is natural for preference
graphs, and preference graphs generalize relevance scores, PD loss is not
directly related to NDCG.

The ranking problem is difficult because the search
space is exponentially large, and the ranking measures and losses are
sensitive only to changes in rank and are hence flat or discontinuous
everywhere (in the output of $h_S$). This makes it difficult to optimize
using standard means like gradient descent. 

It is thus important to find good surrogate losses for these true losses, in
particular ones that can be optimized efficiently for and are statistically
consistent. For the rest of this report, we review consistency results and
algorithms for NDCG and PD loss. 

A side note: since permutations are somewhat awkward, a common method is for
the algorithm to output a scoring function  
$h_S: \bigcup_{n=1}^\infty \X^n \to \mathbb{R}^n$ instead of a function returning
permutations, and then sort with ties broken in some way. In the sequel,
we switch presentation between scores $\vec{s}$ and permutations $\sigma$
as appropriate, e.g., the measures above may be in terms of scores
instead of $\sigma$.

% Surrogates can be classified into being pointwise, pairwise, or
% listwise~\cite{ir-survey}.
% Pointwise surrogates are those of the form
% \begin{align*}
%   \phi(\vec{s}, y) \stackrel{\Delta}{=} \sum_{j=1}^n \phi'(s_j, y_j)
% \end{align*}
% for some $\phi'$, i.e., the loss decomposes over individual objects. 
% Pairwise surrogates are those of the form
% \begin{align*}
%   \phi(\vec{s}, y)
%   &\stackrel{\Delta}{=}
%   \sum_{j,k : i \neq j} \phi'(s_j - s_k, \text{sign}(y_j - y_k)) \\
%   \phi(\vec{s}, G)
%   &\stackrel{\Delta}{=}
%   \sum_{j,k : i \neq j} \phi'(s_j - s_k, w_{j,k}^G) 
% \end{align*}
% for some $\phi'$,
% i.e., the loss decomposes over pairs. Listwise surrogates are
% like the true losses, and do not decompose easily like the other two.
% From their forms, pointwise surrogates apply to
% relevance labels (binary or not), while pairwise and listwise surrogates
% apply to all three forms of labels. Beyond their distinct forms, the more
% salient point is that they exemplify different learning models: a pointwise
% loss determines how a feature vector should be scored; a pairwise
% loss determines how a pair should be scored (positively or negatively), and
% hence induces a ranking 
% 
% \begin{align*}
%   f_S(\vec{x}) = \text{sort}(h_S(\vec{x})) \\
%   \ell(f_S(\vec{x}), y) = \ell(\text{sort}(h_S(\vec{x})), y)
% \end{align*}
% where ties are broken in some way. But the presence of \text{sort} and the 
% But optimizing these ranking measures and losses directly is difficult
% because the losses are 
% 
% Solving the ranking problem directly is difficult because the set of
% permutations is exponentially large. 
% 
% Indeed, several attempts to solve the ranking problem treated
% it in terms of classification and regression; 

The rest of the paper is organized as follows.
In Section~\ref{sec:binary-relevance}, we consider surrogates for losses
defined for binary relevance labels.
In Section~\ref{sec:ndcg}, we consider surrogates for NDCG.
In Section~\ref{sec:pd}, we consider surrogates for PD loss.
