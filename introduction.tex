\section{Introduction to Ranking}
In this project, we study the general problem of supervised ranking and survey
work in the area, paying attention to consistency results and
algorithms that perform empirically well but lack theoretical backing.

Roughly, the goal of supervised ranking is to learn from appropriate samples
in some way so as to learn how to rank, i.e., given a new set of $n$ items,
determine how they can be ordered so as to do well on some performance measure.

The canonical application is web search. Here, the search engine is given
a query $q$ and may find $n$ documents $v_1, \ldots, v_n$, but
has to determine an order in which they should be presented to a user.
Early on, this ordering was determined by hand-tuned models~\cite{bm25},
but there is good reason to believe that models adapting to collected data
(samples) would perform better. 

A sample is a (multi)set of instance-label pairs. Some appropriate types of samples are:
\begin{itemize}
\item (Binary relevance)
  $S = \{ ((q_i, (v_{i, j})_{j=1}^{n}), \y_i )\}_{i=1}^m$, where
  $\y_{i} \in \{0, 1\}^n$.
  That is, an instance is a query $q_i$ with a list of $n$ documents;
  the label for the instance is a vector $\y_i = (y_{i, j})_{j=1}^{n}$ indicating,
  for each document, whether it is relevant or not.
  This may be determined, e.g., by whether there is a
  user who clicked on the link or not.
\item (Relevance scores)
  $S = \{ ( (q_i, (v_{i, j})_{j=1}^{n}), \y_i )\}_{i=1}^m$,
  where $\y_i \in R^n \subseteq \R_+^n$.
  Same as the previous case, except that each document has a relevance
  score. 
  This may be determined, e.g., by the number of users who clicked on the link.
\item (Preference graphs) 
  $S = \{((q_i, (v_{i,j})_{j=1}^{n}), G_i)\})_{i=1}^m$, where $G_i$ is a weighted
  directed graph on the $n_i$ documents, all weights being non-negative.
  We can interpret $v_j \overset{w_{j,k}}{\to} v_k$ as saying that $v_j$ is
  preferred to $v_k$ with an importance weight of $w_{j,k}$.
  Note that this case subsumes the first two: a label for binary relevance
  can be coded as a directed bipartite graph in the obvious way
  with all weights $1$, and a label for relevance scores may be coded as a
  DAG with weights being the difference in scores.
  An important special case is when the graph consists
  of just a single edge, expressing a single pairwise preference.
  Preference graphs allow for cases where complete information on relevance is
  hard to obtain; for instance, an e-commerce ranking system may obtain $S$
  from a consumer survey, and surveyees should be allowed to specify
  preferences for only a small set of items. Ranking from preference graphs
  also have applications beyond just information retrieval, e.g., in
  recommendation systems and social choice theory, where we may want to
  aggregate preferences from different agents into a global ranking~\cite{}.
\end{itemize}
One may view relevance scores as a natural multiclass generalization of binary
relevance labels, and preference graphs as the
structured-prediction-generalization of relevance scores.

In each case, the goal is for the learning algorithm to learn from these samples
and output a ranking function
$f: Q \times D^n \to S_n$, where $Q$
is the set of queries, $D$ is the set of documents,
and $S_n$ is the set of permutations on $[n]$.
The formalism can be simplified by thinking of query-document pairs as feature
vectors in some space $\X$ (which is also done in practice), so
that dependence on queries can be dropped.

Formally, the supervised ranking problem can be framed as follows. There is an
instance space $\X$, a label space $\Y$ (which varies as above), and the
prediction space is $\Yh = S_n$. A permutation
$\sigma \in \Yh$ is interpreted such that $\sigma(i)$ is the position of the
$i$th item. A learning algorithm
takes $S = (\x_i, y_i)_{i=1}^m$ as input,
where $\x_{i} \in \X \subseteq \R^d$ is a feature vector
and $y_i \in \Y$ is a label following the cases above,
and outputs a function
$f_S: \X \to S_n$.
There is no loss of generality, since the feature space can in principle 
include $Q$ itself.

Performance of the learning algorithm is measured by some performance measure
$M: \Yh \times \Y \to \R_+$ or loss function
$\ell: \Yh \times \Y \to \R_+$. The goal is to minimize the $\ell$-risk
\begin{align*}
  \err_D^\ell[f_S] = \E_{({X}, Y) \sim D} [\ell(f_S({X}), Y)],
\end{align*}
for some unknown distribution $D$ on $\X \times \Y$ from which the samples
are drawn i.i.d.
The Bayes $\ell$-risk for a given distribution $D$ is the minimum possible
$\ell$-risk, denoted $\err^{l,*}_D = \inf_{f:\X \to \Yh} \err^l_D[f]$.
A learning algorithm is said to be $\ell$-consistent if for any distribution $D$,
\begin{align*}
  \err_D^l[f_S] \xrightarrow{\Pr} \err_D^{l,*}
  \quad \text{as} \quad
  m \to \infty,
\end{align*}
where $m = |S|$ is the sample size and the probability is over samples $S$ of
size $m$ which is drawn i.i.d. from $D$. Typically, minimizing the empirical
loss $\sum_{i=i}^m \ell(f(\x_i),y_i)$ over some suitable function class $\F_m$
gives a consistent algorithm but is computationally hard.
Therefore surrogate losses are used. Instead of learning a function from
$\X \to \Yh$, one typically learns a function $h_S:\X \to \C$ where the
surrogate space $\C$ is a convex set in $\R^d$ for some $d$. In the case of
ranking, $\C$ is usually $\R^n$ or $\R^{(n)(n-1)/2}$ which correspond to learning
pointwise and pairwise scores respectively.
The learned function is composed with a prediction map $\pred: \C \to \Yh$ to
map instances to permutations, $f_S = \pred \circ h_S$.

The function $h_S$ is obtained by minimizing a surrogate loss $\psi:\C \times \Y \to \R_+$ on the samples in $S$, $h_S \in \argmin_{h\in\H_m}\sum_{i=1}^m \psi(h(\x_i),y_i)$. This gives consistency w.r.t. $\psi$. Given a distribution $D$ over $\X \times \Y$, define $\psi$-risk as follows:
$$\err_D^\psi[h_S] = \E_{(X,Y)\sim D}[\psi(h_S(X),Y)]$$
The learning algorithm is $\psi$-consistent if $\err_D^\psi[h_S]$ converges in probablity to $\err_D^{\psi,*}$ as $m = |S|$ goes to infinity, where $\err_D^{\psi,*} = \inf_{h:\X \to \C}\err_D^\psi[h]$. A surrogate $\psi$ along with a prediction mapping $\pred$ is said to be $\ell$-consistent if $\psi$-consistency of a learning algorithm that outputs $h_S$ on input $S$ implies $\ell$-consistency of the algorithm that ouputs $f_S = \pred \circ h_S$ on input $S$.

Ramaswamy and Agarwal \cite{ramaswamy2016convex} showed that a property of
surrogates called $\ell$-calibration is sufficient for consistency of general
multiclass losses by extending the result for 0-1 loss in
\cite{tewari2007consistency}. They also define the notion of Convex Calibration
dimension which is the smallest $d$ for which there is a calibrated convex
surrogate with a surrogate space $\C \subseteq \R^d$. For most ranking
problems, this is bounded above by $n$ or $n^2$. 

There are many surrogates that are used in practice that are non-convex.
In such cases, one has to analyze the consistency and also the computational
complexity of empirical loss minimization. Some of the surrogates used in
practice are proven to be consistent whereas some are proven to be
inconsistent. There are a few practical approaches for which there are no
consistency results giving opportunity for further work in this area.
The surrogates for some the loss functions studied can be classified as
pointwise, pairwise and listwise. We do not categorize the surrogates
presented in this article. More information can be found in \cite{li2011short}.

% Some common measures and losses are:
% \begin{itemize}
% \item Precision: For binary relevance labels, 
%   \begin{align*}
%     \text{AvgPrec}@K(\sigma, (y_j)_{j=1}^n)
%     = \frac{1}{K} \sum_{r=1}^K \text{Prec}@r(\sigma, (y_j)_{j=1}^n)
%     = \frac{1}{K} \sum_{r=1}^K \frac{\sum_{j : \sigma(j) \leq r} y_j}{r}.
%   \end{align*}
%   Precision at $r$ is the proportion of relevant documents among those 
%   ranked up to $r$ according to $\sigma$. Average precision is the
%   non-truncated version of this measure, and mean average precision (MAP)
%   averages \text{AvgPrec} over all queries in a set, i.e.
%   \begin{align*}
%     \text{MAP}(\sigma, ((y_j)_{j=1}^{n_i})_{i=1}^m) =
%     \frac{1}{m} \sum_{i=1}^m \text{AvgPrec}(\sigma, (y_j)_{j=1}^{n_i}).
%   \end{align*}
% \item Normalized Discounted Cumulative Gain (NDCG): For relevance scores,
%   the discounted cumulative gain at $K$ is 
%   \begin{align*}
%     \text{DCG}@K(\sigma, (y_j)_{j=1}^n)
%     = \sum_{r=1}^K \frac{G(y_{\sigma^{-1}(r)})}{D(r)}
%     = \sum_{i: \sigma(i) \leq K} \frac{G(y_i)}{D(\sigma(i))}.
%   \end{align*}
%   (The first sum runs over positions and the second sum runs over
%   object indices.) The gain function $G$ boosts relevance scores, and is
%   normally taken to be $G(y) = 2^y - 1$. The discount function $D$ discounts
%   the importance of the term based on the position of the item, and is
%   normally taken to be $D(\sigma(i)) = \log(1 + \sigma(i))$. As before, DCG
%   is the non-truncated version of this, and NDCG normalizes DCG to obtain
%   a number in $[0,1]$, defined as
%   \begin{align*}
%     \text{NDCG}(\sigma, (y_j)_{j=1}^n)
%     = \frac{1}{N((y_j)_{j=1}^n)} DCG(\sigma, (y_j)_{j=1}^n),
%   \end{align*}
%   where
%   \begin{align*}
%     N((y_j)_{j=1}^n)
%     = \frac{1}{\max_{\sigma'} DCG(\sigma', (y_j)_{j=1}^n)} DCG(\sigma, (y_j)_{j=1}^n).
%   \end{align*}
%   In many applications, it is more important for the output to rank well for
%   items that are near the top compared to ones below, and to rank well for
%   highly relevant items compared to those with low scores; NDCG factors
%   these in using the discount and gain functions respectively.
% 
% \item Weighted pairwise disgreement loss: For preference graphs, the most
%   natural loss to define is just:
%   \begin{align*}
%     \ell_{\text{PD}}(\sigma, G)
%     = \sum_{j,k: j \neq k} w_{j,k}^G \1[\sigma(j) > \sigma(k)]
%   \end{align*}
%   Here the sum is over object indices, and we can take $w_{j,k}^G = 0$ if
%   the edge $j \to k$ is not in $G$. 
% \end{itemize}
% Note that while the pairwise disagreement (PD) loss is natural for preference
% graphs, and preference graphs generalize relevance scores, PD loss is not
% directly related to NDCG.
% 
% The ranking problem is difficult because the search
% space is exponentially large, and the ranking measures and losses are
% sensitive only to changes in rank and are hence flat or discontinuous
% everywhere (in the output of $h_S$). This makes it difficult to optimize
% using standard means like gradient descent. 
% 
% It is thus important to find good surrogate losses for these true losses, in
% particular ones that can be optimized efficiently for and are statistically
% consistent. For the rest of this report, we review consistency results and
% algorithms for NDCG and PD loss. 
% 
% A side note: since permutations are somewhat awkward, a common method is for
% the algorithm to output a scoring function  
% $h_S: \bigcup_{n=1}^\infty \X^n \to \mathbb{R}^n$ instead of a function returning
% permutations, and then sort with ties broken in some way. In the sequel,
% we switch presentation between scores $\vec{s}$ and permutations $\sigma$
% as appropriate, e.g., the measures above may be in terms of scores
% instead of $\sigma$.

% Beyond their distinct forms, the more salient point is that they exemplify
% different approaches to the problem.
% A pointwise loss is about determining the rank of an item directly.
% When training, a sample point $((x_{i,j})_{j=1}^{n_i}, y_i)$ may be broken up
% into sample points $\{(x_{i, j}, y_{i, j})\}_{j=1}^{n_i}$, and the loss may be
% minimized using techniques from ordinal regression~\cite{}.
% A pairwise loss is about determining when an item should be preferred to
% another. When training, a sample point $((x_{i,j})_{j=1}^{n_i}, y_i)$ may be
% broken up into sample points
% $\{ (x_{i, j_1}, x_{i, j_2}, y_{i, j_1}, y_{i, j_2}) \}_{j_1, j_2 \in [1, n_i]: j_1 \neq j_2}$.
% Suitable losses can be based on max-margin
% principles~\cite{ranking-with-large-margin-principle} or 
% some probability model of $\Pr[\text{$x_1$ is ranked higher than $x_2$}]$,
% e.g. KL-divergence~\cite{ranknet}. 
% A listwise loss is about determining the full ranking directly. Suitable losses
% may also be based on probabilistic models, but of
% $\Pr[\sigma \mid y = (y_{i,1} \ldots y_{i, n_i})]$ and/or
% $\Pr[\sigma \mid s = (s_{i,1} \ldots s_{i, n_i})]$~\cite{listmle, listnet}.
% Pairwise and listwise approaches outperform pointwise approaches~\cite{letor},
% intuitively because they exploit structure in the samples. For this reason, 
% we focus on pairwise and listwise approaches, and review consistency results
% and algorithms for binary relevance, NDCG and PD losses in the following
% sections.

The rest of the report is organized as follows.
In Section~\ref{sec:loss0-1} we look at the standard 0-1 loss on permutations.
In Section~\ref{sec:binary-relevance}, we consider surrogates for losses
defined for binary relevance labels.
In Section~\ref{sec:ndcg}, we consider surrogates for NDCG.
In Section~\ref{sec:pd}, we consider surrogates for PD loss.
In Section~\ref{sec:problems}, we look at extensions to ranking and other
problems considered in this project.
