\section{Subset Ranking}\label{sec:binary-relevance}
Binary relevance, also known as subset ranking, is the case where the true labels are subsets of $[n]$, i.e., $\Y = \{0,1\}^n$. There are various loss functions in this setting most of which are low rank meaning that the loss function $\ell:\Yh \times \Y \to \R_+$ when viewed as a $|\Yh|\times|\Y|$ matrix has small rank. Hence the quadratic surrogate loss for low rank losses defined in \cite{ramaswamy2013convex} can be applied directly to get consistent algorithms for these problems. Here, we look at a few important loss functions.

\subsection{Precision@$q$}
The precision@$q$ loss is a measure of how (im)precise the prediction is, with respect to the true label when we look at the prediction $\sigma$ as a map from $[n] \to \{0,1\}$ which sends the top $q$ positions to $1$ and the rest to $0$. More precisely, the loss $\ell_{P@q}: \{0,1\}^n \times S_n\to \R_+$ is defined by,
$$\ell_{P@q}(\sigma,\y) = 1 - \frac{1}{q}\sum_{i:\sigma(i)\leq q}y_i$$
A consistent surrogate loss $\psi_{P@q}:\R^n \times \Y \to \R_+$ and its corresponding prediction mapping $\pred_{P@q}:\R^n \to S_n$ are given by:
$$\psi_{P@q}(\u,\y) = \lVert \u - \y \rVert_2^2$$
$$\pred_{P@q}(\u) \in \argmax_{\sigma\in S_n}\sum_{i:\sigma(i)\leq q}u_i$$
In this case, the prediction mapping just amounts to sorting the indices in decreasing order of the $\u$ values. 

\subsection{MAP}
The mean average precision, as the name suggests, is a loss based on the avergae of precision at positions where the permutation maps the documents labelled 1 in $\y$. The loss $\ell_{\text{MAP}}:\{0,1\}^n \times S_n \to \R_+$ is given by,
$$\ell_{\text{MAP}} = 1 - \frac{1}{\lVert\y\rVert_1}\sum_{i:y_i = 1}\frac{1}{\sigma(i)}\sum_{j:\sigma(j) \leq \sigma(i)}y_j$$
where $\lVert\y\rVert_1 = \sum_{i=1}^ny_i$ is the number relevant documents according to $\y$. It is known that there cannot be an $n$-dimensional convex calibrated surrogate for this loss. \cite{ramaswamy2013convex} gives an $O(n^2)$ dimensional surrogate for this loss. The consistent (convex) surrogate $\psi_{\text{MAP}}:  \R^{n(n+1)/2} \times \{0,1\}^n \to \R_+$ and the corresponding $\pred$ mapping is as follows:
$$\psi_{\text{MAP}}(\u,\y) = \sum_{i=1}^n\sum_{j=1}^i\Big(u_{ij} - \frac{y_iy_j}{\lVert\y\rVert_1}\Big)^2$$
$$\pred_{\text{MAP}(\u)} \in \argmax_{\sigma\in S_n}\sum_{i=1}^n\sum_{j=1}^i \frac{u_{ij}}{\max(\sigma(i),\sigma(j))}$$
This method involves learning a score for every unordered pair of documents/indices where a higher score indicates that both elements of the pair should be higher in rank. Unfortunately, there is no known polynomial time algorithm to compute the $\pred$ function. Therefore, one has to make lwo noise assumptions on the distribution $D$ to get fast algorithms. Under one such condition which assumes that the expected values of $y_{ii}/\lVert\y\rVert_1$ given any $x \in \X$ are more relevant than the cross terms $y_{ij}/\lVert\y\rVert_1$ where $i \neq j$, one can simply sort the indices in decreasing order of the diagonal terms $u_{ii}$ to compute the permutation from scores. In that scenario, it is also enough to just learn the diagonal terms giving an $n$-dimensional surrogate.  